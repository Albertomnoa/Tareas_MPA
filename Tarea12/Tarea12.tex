\documentclass{article}
\usepackage[utf8]{inputenc}
\usepackage[spanish]{babel}
\usepackage{listings}
\usepackage{subfigure}
\usepackage{graphicx}
\usepackage{url}
\usepackage{multirow}
\usepackage{multicol}
\usepackage{color}
\usepackage{booktabs}
\usepackage{float}
\usepackage{amsmath,amssymb,amscd,amsthm}
%\usepackage{verbatim}
\usepackage{hyperref}
\hypersetup{
    colorlinks=true,
    linkcolor=blue,
    filecolor=magenta,      
    urlcolor=cyan,
}
\usepackage[margin=3cm,twoside]{geometry} 
\setlength{\parindent}{0pt}
\setlength{\parskip}{1em}
\usepackage{fancyvrb}
\usepackage{enumerate}
\newcommand\ql{\textquotedblleft}
\newcommand\qr{\textquotedblright}


\newcommand\E{\ensuremath{\mathbb{E}}}
\newcommand\N{\ensuremath{\mathbb{N}}}
\renewcommand{\P}{\ensuremath{\mathbb{P}}}
\newcommand\Q{\ensuremath{\mathbb{Q}}}
\newcommand\R{\ensuremath{\mathbb{R}}}
\newcommand\Z{\ensuremath{\mathbb{Z}}}

\newcommand\Om{\ensuremath{\Omega}}
\newcommand{\w}{\ensuremath{\omega}}

\newcommand\var[1]{\, \mathrm{Var} \left( #1 \right)}

\newcommand\pr[1]{\, \mathbb{P} \left( #1 \right)}

\newcommand\cov[1]{\, \mathrm{Cov} \left( #1 \right)}

\newcommand\espe[1]{\, \mathbb{E} \lbrack #1 \rbrack}



\definecolor{mygreen}{rgb}{0,0.6,0}
\definecolor{mygray}{rgb}{0.5,0.5,0.5}
\definecolor{mymauve}{rgb}{0.58,0,0.82}
\lstset{ 
  backgroundcolor=\color{white},   % choose the background color; you must add \usepackage{color} or \usepackage{xcolor}; should come as last argument
  basicstyle=\footnotesize,        % the size of the fonts that are used for the code
  breakatwhitespace=false,         % sets if automatic breaks should only happen at whitespace
  breaklines=true,                 % sets automatic line breaking
  captionpos=b,                    % sets the caption-position to bottom
  commentstyle=\color{mygreen},    % comment style
  deletekeywords={...},            % if you want to delete keywords from the given language
  escapeinside={\%*}{)},          % if you want to add LaTeX within your code
  extendedchars=true,              % lets you use non-ASCII characters; for 8-bits encodings only, does not work with UTF-8
  firstnumber=1,                % start line enumeration with line 1000
  frame=single,	                   % adds a frame around the code
  keepspaces=true,                 % keeps spaces in text, useful for keeping indentation of code (possibly needs columns=flexible)
  keywordstyle=\color{blue},       % keyword style
  language=Octave,                 % the language of the code
  morekeywords={*,...},            % if you want to add more keywords to the set
  numbers=left,                    % where to put the line-numbers; possible values are (none, left, right)
  numbersep=5pt,                   % how far the line-numbers are from the code
  numberstyle=\tiny\color{mygray}, % the style that is used for the line-numbers
  rulecolor=\color{black},         % if not set, the frame-color may be changed on line-breaks within not-black text (e.g. comments (green here))
  showspaces=false,                % show spaces everywhere adding particular underscores; it overrides 'showstringspaces'
  showstringspaces=false,          % underline spaces within strings only
  showtabs=false,                  % show tabs within strings adding particular underscores
  stepnumber=1,                    % the step between two line-numbers. If it's 1, each line will be numbered
  stringstyle=\color{mymauve},     % string literal style
  tabsize=2,	                   % sets default tabsize to 2 spaces
  title=\lstname                  % show the filename of files included with \lstinputlisting; also try caption instead of title
}
\usepackage{etoolbox}
\makeatletter
\providecommand{\subtitle}[1]{% add subtitle to \maketitle
  \apptocmd{\@title}{\par {\large #1 \par}}{}{}
}
\RecustomVerbatimCommand{\VerbatimInput}{VerbatimInput}%
{fontsize=\footnotesize,
	%
	frame=lines,  % top and bottom rule only
	framesep=2em, % separation between frame and text
	rulecolor=\color{mygreen},
	%
	label=\fbox{\color{black} prueba.txt},
	labelposition=topline,
	%
	commandchars=\|\(\), % escape character and argument delimiters for
	% commands within the verbatim
	%commentchar=*        % comment character
}
\renewcommand{\theenumi}{\roman{enumi}}
\newtheorem{teor}{Teorema}
\makeatother
 \title{Tarea 12 de Modelos Probabilistas Aplicados}
\subtitle{Ejercicios}
\author{5271}
\date{\today}

\begin{document}

\maketitle
\section{Introducción}
En este documento se presentan los resultados de varios ejercicios del libro ``\textit{Introduction to Probability}''\cite{libProba}.
\section{Ejercicio 1 de la página 392}
\textit{Let $Z_1, Z_2, \ldots, Z_n$ describe a branching process in which each parent has $j$ offspring with probability $p_j$. Find the probability $d$ that the process eventually dies out if}
\begin{enumerate}[a)]
				\item $p_0 = \frac{1}{2}, p_1 = \frac{1}{4}, p_2 = \frac{1}{4}$.
				\item $p_0 = \frac{1}{3}, p_1 = \frac{1}{3}, p_2 = \frac{1}{3}$.
				\item $p_0 = \frac{1}{3}, p_1 = 0, p_2 = \frac{2}{3}$.
				\item $p_j = \frac{1}{2}^{j + 1}$, for $j = 0, 1, 2,... $
				\item $p_j = (\frac{1}{3}) (\frac{2}{3})^j$, for $j = 0, 1, 2,... $
				 \item $p_j = e^{-2} 2^j / j!$, for $j = 0, 1, 2,... $ (estimate $d$ nummerically)
			\end{enumerate}
De acuerdo al teorema 10.3 si el número medio m de descendientes producidos por un solo padre es $\leq 1$, entonces d $= 1$ y el proceso se extingue con probabilidad uno. Si m $> 1$ entonces d $< 1$ y el proceso se extingue con probabilidad d. 
Para la resolución de los incisos a), b), c) se utiliza la expresión siguiente:
\begin{equation}
\mathrm{m}  = p_{1}+2p_{2} = 1-p_{0}+p_{2}    
\end{equation}
\paragraph{a)}$p_0 = \frac{1}{2}, p_1 = \frac{1}{4}, p_2 = \frac{1}{4}$.

 \begin{align}
   \mathrm{m}  &= p_{1}+2p_{2} \\
               &= \frac{1}{4} + 2\left(\frac{1}{4}\right)\\
               &= \frac{3}{4}
 \end{align}
Por tanto m es menor que uno y la probabilidad de decadencia es uno.
 
\paragraph{b)}$p_0 = \frac{1}{3}, p_1 = \frac{1}{3}, p_2 = \frac{1}{3}$.
\begin{align}
   \mathrm{m}  &= p_{1}+2p_{2} \\
               &= \frac{1}{3} + 2\left(\frac{1}{3}\right)\\
               &= 1
 \end{align}
 Por tanto m es igual a uno y la probabilidad de decadencia es uno.
 \paragraph{c)}$p_0 = \frac{1}{3}, p_1 = 0, p_2 = \frac{2}{3}$.
 
 \begin{align}
   \mathrm{m}  &= p_{1}+2p_{2} \\
               &= 0 + 2\left(\frac{2}{3}\right) \\
               &= \frac{4}{3}
 \end{align}
 Por tanto m es mayor que uno y la probabilidad de decadencia es igual a d. A contención pasamos a calcular la probabilidad de decadencia (d) 
 \begin{align} 
        d & = \frac{p_{0}}{p_{2}}\\ 
          & = \frac{\frac{1}{3}}{\frac{2}{3}}\\
        d & = \frac{1}{2} \\ 
\end{align}
\paragraph{d)}$p_j = \frac{1}{2}^{j + 1}$, for $j = 0, 1, 2, $.
 \begin{align}
    h(z) &= p_0 + p_1z + p_2z^2 + p_3z^3 +p_4z^4 ... \\
      &= \frac{1}{2}^{0 + 1} + \frac{1}{2}^{1 + 1}z + \frac{1}{2}^{2 + 1}z^2 + \frac{1}{2}^{3 + 1}z^3 +\frac{1}{2}^{4 + 1}z^4 +...\\
      &= \frac{1}{2}^{1} + \frac{1}{2}^{2}z + \frac{1}{2}^{3}z^2 + \frac{1}{2}^{4}z^3 +\frac{1}{2}^{5}z^4 .. \\
      &= \frac{1}{2} (1 + 1 / 2^{1}z + 1 / 2^{2}z^2 + 1 / 2^{3}z^3 +...)\\
      &= \frac{1}{2} \left(\frac{1}{1 - \frac{1}{2} z}\right)\\
      &= \frac{1}{2 - z}.
  \end{align}
 Calculando  h'(z) tenemos:
 \begin{align} 
        h'(z) &= \frac{d}{dz} \left( \frac{1}{2 - z} \right)\\
         &= \frac{- \frac{d}{dz} \left( 2 - z \right)}{(2 - z)^2}\\
          &= \frac{1}{(2 - z)^2}\\
\end{align}
ahora calculamos h'(1):
 $m = h'(1) = \frac{1}{(2 - 1)^2} = 1 $
 Como m es igual a 1 entones la probabilidad de decadencia es uno.

\section{Ejercicio 3 de la página 392}	
\textit{In the chain letter problem (see Example 10.14) find your expected profit if}

			\begin{enumerate}[a)]
				\item $p_0 = 1/2, \, p_1 = 0, \, p_2 = 1/2$.
				\item $p_0 = 1/6, \, p_1 = 1/2, \, p_2 = 1/3$.
			\end{enumerate}
		\textit{	Show that if} $p_0 > 1 / 2$,\textit{ you cannot expect to make a profit.}
		
Para la resolver este ejercicio se se emplea la siguiente expresión: 
\begin{equation}
    50\mathrm{m} + 50 \mathrm{m}^{12}, \quad \text{con}\quad  \mathrm{m} = p_1 + 2p_2.
\end{equation}

	
\paragraph{a)} $p_0 = 1/2, \, p_1 = 0, \, p_2 = 1/2$.
 Sustituyendo en la expresión anterior se tiene:
 \begin{align}
     	\mathrm{m} &=0 + 2\left(\frac{1}{2}\right)\\
     	   &=1,
 \end{align}
 por lo que:
  \begin{align}
  50(1)+50(1^{12})-100 =0   
 \end{align}
 Por tanto el la ganancia esperada es cero.

\paragraph{b)} $p_0 = 1/6, \, p_1 = 1/2, \, p_2 = 1/3$.
	 \begin{align}
     	\mathrm{m} &=\frac{1}{2} + 2\left(\frac{1}{3}\right)\\
     	   &=\frac{7}{6},
 \end{align}
 por lo que:
  \begin{align}
  50\left(\frac{7}{6}\right)+50\left(\frac{7}{6}\right)^{12}-100 \approx 276.3
 \end{align}
 Por tanto el la ganancia esperada es 276.3.
 
Para todos los posibles valores de $p_0 > 1 / 2 $, $p_2<p_0$ por lo tanto  $\mathrm{m}<1$ y $50\mathrm{m} + 50 \mathrm{m}^{12}<100$, por lo que el juego no va a ser favorable.
\section{Ejercicio 1 de la página 401 }
\textit{Let $X$ be a continuous random variable with values in $[0,2]$ and density $f_X$. Find the moment generating function $g(t)$ for $X$ if}
\begin{enumerate}[a)]
  \item $f_X(x) = \frac{1}{2}$.
  \item $f_X (x) = \frac{1}{2}x$.
  \item  $f_X (x) = 1 - \frac{1}{2}x$.
  \item $f_X (x) = |1 - x|$.
  \item $f_X (x) = \frac{3}{8}x^2$.
  \end{enumerate}
   Para la realización de este ejercicio se utiliza la ecuación siguiente:
  \begin{align}
    g(t)&=\espe{e^{tx}}\\
       &= \displaystyle\int_{-\infty}^\infty e^{tx}fX(x)\text{d} x
  \end{align}   
     

 
\paragraph{a)}
$fX(x) = \frac{1}{2}.$

\begin{align}
   g(t) &= \displaystyle\int_{0}^2 {e}^{tx}\cdot\dfrac{1}{2}{\mathrm{d}x}   \\
        &=    {\displaystyle\int}\dfrac{\mathrm{e}^{tx}}{2}\,\mathrm{d}x\\
        &= \dfrac{1}{2t}{\displaystyle\int}\mathrm{e}^u\,\mathrm{d}u \\       
        &=  \dfrac{\mathrm{e}^{tx}}{2t} \\
        &=  \dfrac{\frac{\mathrm{e}^{2t}}{t}-\frac{1}{t}}{2}  \\
        &=   \dfrac{\mathrm{e}^{2t}-1}{2t}.
\end{align}
\paragraph{b)}
$fX(x) = \frac{1}{2}(x).$

\begin{align}
   g(t) &= \displaystyle\int_{0}^2 {e}^{tx}\cdot\dfrac{1}{2}x{\mathrm{d}x}   \\
        &=  \dfrac{1}{2}{\displaystyle\int}x\mathrm{e}^{tx}\,\mathrm{d}x.
\end{align}
   Resolviendo: 
    \begin{equation*}
        {\displaystyle\int}x\mathrm{e}^{tx}\,\mathrm{d}x,\\
    \end{equation*}
    se tiene:
        \begin{align}
         &= \dfrac{x\mathrm{e}^{tx}}{t}-{\displaystyle\int}\dfrac{\mathrm{e}^{tx}}{t}\,\mathrm{d}x
         \end{align}
          Ahora se resuelve: 
    \begin{equation*}
        {\displaystyle\int}\dfrac{\mathrm{e}^{tx}}{t}\,\mathrm{d}x,\\
    \end{equation*}  
    y se tiene:
   \begin{align}      
        &=  \dfrac{1}{t^2}{\displaystyle\int}\mathrm{e}^u\,\mathrm{d}u \\
        &=  \dfrac{\mathrm{e}^u}{t^2} \\
        &=   \dfrac{\mathrm{e}^{tx}}{t^2}.
\end{align}
Remplazando las integrales resuelta se tiene:
 \begin{align}
 &=\dfrac{1}{2}\left[\dfrac{x\mathrm{e}^{tx}}{t}-\dfrac{\mathrm{e}^{tx}}{t^2}\right]\\
 &= \dfrac{x\mathrm{e}^{tx}}{2t}-\dfrac{\mathrm{e}^{tx}}{2t^2}\\
  &=   \dfrac{\left(tx-1\right)\mathrm{e}^{tx}}{2t^2}\\
  &= \dfrac{\left(2t-1\right)\mathrm{e}^{2t}+1}{2t^2}.
\end{align}

\paragraph{c)}
$fX(x) = 1-\frac{1}{2}(x).$

\begin{align}
   g(t) &= \displaystyle\int_{0}^2 {e}^{tx}\cdot\left(1-\dfrac{1}{2}x\right){\mathrm{d}x}   \\
        &=  -\dfrac{1}{2}{\displaystyle\int}\left(x-2\right)\mathrm{e}^{tx}\,\mathrm{d}x.
\end{align}
   Resolviendo: 
    \begin{equation*}
        {\displaystyle\int}\left(x-2\right)\mathrm{e}^{tx}\,\mathrm{d}x,\\
    \end{equation*}
    se tiene:
        \begin{align}
         &= \dfrac{\left(x-2\right)\mathrm{e}^{tx}}{t}-{\displaystyle\int}\dfrac{\mathrm{e}^{tx}}{t}\,\mathrm{d}x
         \end{align}
          Ahora se resuelve: 
    \begin{equation*}
       {\displaystyle\int}\dfrac{\mathrm{e}^{tx}}{t}\,\mathrm{d}x,\\
    \end{equation*}  
    y se tiene:
   \begin{align}      
        &=  \dfrac{1}{t^2}{\displaystyle\int}\mathrm{e}^u\,\mathrm{d}u \\
        &=  \dfrac{\mathrm{e}^u}{t^2} \\
        &=   \dfrac{\mathrm{e}^{tx}}{t^2}.
\end{align}
Remplazando las integrales resuelta se tiene:
 \begin{align}
 &=-\dfrac{1}{2}\left[\dfrac{\left(x-2\right)\mathrm{e}^{tx}}{t}-\dfrac{\mathrm{e}^{tx}}{t^2}\right]\\
 &= \dfrac{\mathrm{e}^{tx}}{2t^2}-\dfrac{\left(x-2\right)\mathrm{e}^{tx}}{2t}\\
  &=   \dfrac{\mathrm{e}^{2t}}{2t^2}-\dfrac{2t+1}{2t^2}\\
  &= \dfrac{\mathrm{e}^{2t}-2t-1}{2t^2}.
\end{align}

\paragraph{d)}
$fX(x) = \left|x-1\right|.$

\begin{align}
   g(t) &= \displaystyle\int_{0}^2 {e}^{tx}\cdot\left|x-1\right|{\mathrm{d}x}   \\
        &=  \dfrac{\left|x-1\right|\mathrm{e}^{tx}}{t}-{\displaystyle\int}\dfrac{\left(x-1\right)\mathrm{e}^{tx}}{t\left|x-1\right|}\,\mathrm{d}x.
\end{align}
   Resolviendo: 
    \begin{equation*}
        {\displaystyle\int}\dfrac{\left(x-1\right)\mathrm{e}^{tx}}{t\left|x-1\right|}\,\mathrm{d}x,\\
    \end{equation*}
    se tiene:
        \begin{align}
         &= \dfrac{1}{t^2}{\displaystyle\int}1\,\mathrm{d}u,
         \end{align}
          ahora se resuelve: 
    \begin{equation*}
       {\displaystyle\int}1\,\mathrm{d}u,\\
    \end{equation*}  
    y se tiene:
   \begin{align}      
        &= u\\
         &=\dfrac{u}{t^2},
        \end{align}
        desciendo la sustitución:
        \begin{align}  
        &=  \dfrac{\left(x-1\right)\mathrm{e}^{tx}}{t^2\left|x-1\right|}.
\end{align}
Remplazando las integrales resuelta se tiene:
 \begin{align}
 &=\dfrac{\left|x-1\right|\mathrm{e}^{tx}}{t}-\dfrac{\left(x-1\right)\mathrm{e}^{tx}}{t^2\left|x-1\right|}\\
 &= \dfrac{\left(t-1\right)\mathrm{e}^{2t}}{t^2}+\dfrac{2\mathrm{e}^t}{t^2}-\dfrac{t+1}{t^2}\\
  &= \dfrac{\left(t-1\right)\mathrm{e}^{2t}+2\mathrm{e}^t-t-1}{t^2}.
\end{align}

\paragraph{e)}
$fX(x) = \frac{3}{8}(x^{2}).$

\begin{align}
     g(t) &= \displaystyle\int_{0}^2 {e}^{tx}\cdot\dfrac{3}{8}x^{2}{\mathrm{d}x}   \\
        &=  \dfrac{3}{8}{\displaystyle\int}x^2\mathrm{e}^{tx}\,\mathrm{d}x.
\end{align}
   Resolviendo: 
    \begin{equation*}
        {\displaystyle\int}x^2\mathrm{e}^{tx}\,\mathrm{d}x,\\
    \end{equation*}
    se tiene:
        \begin{align}
         &= \dfrac{x^2\mathrm{e}^{tx}}{t}-{\displaystyle\int}\dfrac{2x\mathrm{e}^{tx}}{t}\,\mathrm{d}x,
         \end{align}
          ahora se resuelve: 
    \begin{equation*}
       {\displaystyle\int}\dfrac{2x\mathrm{e}^{tx}}{t}\,\mathrm{d}x,\\
    \end{equation*}  
    y se tiene:
   \begin{align}      
        &= \dfrac{2}{t}{\displaystyle\int}x\mathrm{e}^{tx}\,\mathrm{d}x,
               \end{align}
          resolviendo: 
    \begin{equation*}
       {\displaystyle\int}x\mathrm{e}^{tx}\,\mathrm{d}x,\\
    \end{equation*}  
   se tiene:
   \begin{align} 
         &=\dfrac{1}{t^2}{\displaystyle\int}\mathrm{e}^u\,\mathrm{d}u\\
          &= \dfrac{\mathrm{e}^u}{t^2}\\
          &==\dfrac{\mathrm{e}^{tx}}{t^2}
\end{align}
Remplazando las integrales resuelta se tiene:
 \begin{align}
 &=\dfrac{3}{8}\left[\dfrac{x^2\mathrm{e}^{tx}}{t}-\dfrac{2x\mathrm{e}^{tx}}{t^2}+\dfrac{2\mathrm{e}^{tx}}{t^3}\right]\\
 &= \dfrac{3x^2\mathrm{e}^{tx}}{8t}-\dfrac{3x\mathrm{e}^{tx}}{4t^2}+\dfrac{3\mathrm{e}^{tx}}{4t^3}\\
  &= \dfrac{\left(6t^2-6t+3\right)\mathrm{e}^{2t}-3}{4t^3}.
\end{align}


\section{Ejercicio 6 de la página 402}

\textit{Let $X$ be a continuous random variable whose characteristic function $k_X (\tau)$ is $k_X(\tau) = e^{-|\tau|}$, $-\infty < \tau < \infty$. Show directly that the density $f_X$ of $X$ is}
  \begin{equation*}
      f_X(x) = \frac{1}{\pi (1+x^2)}.
  \end{equation*}
Para la solución de este ejercicio se utilizara la formula que se muestra a continuación:
  \begin{equation}
      f_X(x) = \frac{1}{2\pi}\displaystyle\int_{-\infty}^\infty e^{-i\tau x}\cdot k_{X}(\tau)\mathrm{d}\tau .
  \end{equation}
\paragraph{Respuesta}  sustituyendo en la ecuación anterior se tiene: 
  \begin{align}
      f_X(x) &= \frac{1}{2\pi}\displaystyle\int_{-\infty}^\infty e^{-i\tau x}\cdot e^{-|\tau|}\mathrm{d}\tau \\ 
             &=\frac{1}{2\pi} \left( \int_{-\infty}^{0}e^{-ix\tau} e^{\tau}\, d\tau \right) + \frac{1}{2\pi} \left( \int_{0}^{\infty}e^{-ix\tau} e^{-\tau}\, d\tau \right)  \\
            &= \frac{1}{2\pi} \left(\dfrac{\mathrm{i}x+1}{x^2+1}\right)+\frac{1}{2\pi}\left[- \left(\dfrac{\mathrm{i}x-1}{x^2+1}\right)\right]\\
            &=\frac{1}{2\pi}\left( \dfrac{\mathrm{i}x+1 -\mathrm{i}x +1}{x^2+1} \right)\\
            &= \frac{1}{2\pi}\left( \dfrac{2}{x^2+1} \right)\\
            &= \frac{1}{\pi (1+x^2)}.
  \end{align}

\section{ejercicio 10 de la página 402}

  \textit{Let $X_1, X_2, \ldots, X_n$ be an indepentent trials process with density}
  \begin{equation*}
    f(x) = \frac{1}{2} e^{-|x|}, -\infty < x < + \infty.
  \end{equation*}
  \begin{enumerate}[a)]
    \item Find the mean and variance of $f(x)$.
    \item Find the moment generating function for $X_1, S_n, A_n$, and $S_n^*$.
    \item What can you say about the moment generating function of $S_n^*$ as $n \rightarrow \infty$.
    \item What can you say about the moment generating function of $A_n$ as $n \rightarrow \infty$.
  \end{enumerate}
\paragraph{a)}Find the mean and variance of $f(x)$.

				\begin{align}
				   	\mathbb{E}(X)  & = \int_{-\infty}^{\infty}x\dfrac{e^{-|x|}}{2} dx \\
				& = \dfrac{1}{2}\int_{-\infty}^{\infty}  \dfrac{x}{|x|} e^{-|x|} |x| dx\\
				& = - \dfrac{e^{-|x|} |x|}{2} - \dfrac{e^{-|x|}}{2}\\
				& = \left[ \dfrac{e^{-|x|} (-|x|-1)}{2}\right] _{-\infty}^{\infty}\\
				& = 0. 
				\end{align}
				
				Ya se tiene el valor de $\espe{X}$, pasamos a calcular $\mathbb{V}(X)$ con la expresión:
				\begin{equation}
				    	\mathbb{V}(X) = \espe{X^{2}} - \espe{X}^{2})\\
				\end{equation}
		
			\begin{align}
			   \espe{X}^{2}) & =  \int_{-\infty}^{\infty} x^2 \frac{1}{2}e^{-|x|} \, dx \\ 
                    & = \frac{1}{2} \left[\int_{-\infty}^{0} x^2 e^{-(-x)} \, dx + \int_{0}^{\infty} x^2 e^{-(x)} \, dx \right] \\ 
                    & = \frac{1}{2} \left[\int_{-\infty}^{0} x^2 e^{(x)} \, dx + \int_{0}^{\infty} x^2 e^{-(x)} \, dx \right] \\ \nonumber
                    & = \frac{1}{2} \left[2+2 \right] \\ 
                    & = 2. 
			\end{align}
		Entonces: 
		 \begin{align}
      \mathbb{V}(X)  & = \espe{X^{2}} - \espe{X}^{2})\\
        & = 2 - (0)^2 \\ 
                & = 2 - 0 \\
                & = 2. 
    \end{align}
   
\newpage
\bibliographystyle{plain}
\bibliography{Biblio}

\end{document}
